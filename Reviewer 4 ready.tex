\documentclass{article}
\usepackage{graphicx} % Required for inserting images
\usepackage{enumitem}
\usepackage{color}
\usepackage{booktabs}
\usepackage[table,xcdraw]{xcolor}
\usepackage{amsmath, amsthm, amssymb}

\newcommand{\textchange}[1]{\textcolor{blue}{#1}}

\title{Enhanced-Beacons Dynamic Transmission over TSCH} %Response to the  Editor and the Reviewers of Paper futureinternet-2995965}
%\author{Erik Ortiz Guerra, Mario Mart\'inez Morfa, Carlos Manuel Garc\'ia Algora,\\ Hector Cruz Enriquez, Kris Steenhaut, and Samuel Montejo-S\'anchez }
\date{May 2024}

\begin{document}

\maketitle
\begin{center}
\textbf{Response to the  Editor and the Reviewers \\ \medskip Manuscript futureinternet-2995965 \\ \medskip} 

\end{center}

We are grateful to the Editor and the anonymous Reviewers for their valuable time and for providing many constructive comments and suggestions, which helped us to improve the manuscript. After being carefully and thoroughly revised, the manuscript has undergone several modifications, for instance:

\begin{itemize}
     \item The abstract was rewritten.
     \item The paper contributions were detailed at the end of section 2. Likewise, throughout the text, the distinctive characteristics of the proposed mechanism were highlighted.
     \item Equations (3), (6), and (A.1) were corrected by replacing ``the elapsed time to send $j$ EBs, $t_j$'' with ``the average elapsed time to send $j$ EBs, $E(t_j)$''.
     \item The parameters used for the simulations were better described; for this, Table 1 was included, which details each parameter. A description of the hardware used to run the simulator was also added.
     \item The term ``distance from coordinator node'' used in Figure 9 was properly defined.
     \item The English was checked and the typos and errors found were corrected.
     \item Conclusions were rewritten and future work was added.
     \item To enrich the discussion on the state-of-the-art and describe the application scenarios of the proposed mechanism, the following bibliographical references were added: [4], [6], [7], [32-36].
\end{itemize}

In the revised manuscript, we have highlighted the changes in \textchange{blue} for your convenience. We hope the Editor and the Reviewers are satisfied with our replies and modifications in the manuscript, which are detailed next.

The authors

\newpage
\section*{\centering Response to the Comments of Reviewer 4}

\textit{The paper proposes an enhanced-beacons dynamic transmission method to enhance the time-slotted channel hopping (TSCH) protocol. The problem has high relevance in the domain of Internet of Things where TSCH is used as a standardised operational mode of medium access control.}

\textit{The proposed method improves the synchronization phase of the protocol. Although the proposed improvement is quite simple, the paper has significant research value because it analyses the method both theoretically and experimentally in detail. The evaluations show that the method reduces the time of network formation and the energy consumption compared with other methods.}

\textit{I have the following comments and recommendations:}
\\

We greatly appreciate these constructive comments and favorable opinions regarding our work. We wanted to let you know that his/her report arrived in the system after the major revision notification, so it wasn't until now that we saw it. Despite this, we respond to his/her concerns in the following.

\begin{enumerate}

\item \textbf{Reviewer comment:} \textit{In the evaluation, the proposed method is compared with minimal 6TiSCH only. However, the Subsection Related work presents several methods to reduce the network formation time. Why do not you compare the proposed method with other state-of-the-art methods? How can we know that the proposed method performs really better than other methods in the literature?}

\textbf{Answer:} We thank the reviewer for his/her thoughtful questions and further agree with the usefulness of comparisons with SOTA mechanisms. Unfortunately, most state-of-the-art works are not explicit enough to guarantee their reproducibility, others are developed for very different scenarios, so their direct comparison would not be fair. Therefore, we consider it most appropriate to compare our proposal with the default minimal 6TiSCH configuration, since comparisons with this benchmark allow for a uniform and fair comparison of the different new proposals.

Section 5 is dedicated to evaluating the proposed EBDT-TSCH mechanism through simulation and comparing it with the default version of 6TiSCH. In each subsection, the scenario used is described and the results achieved are discussed. The discussion of the results, not only describes the behavior of the curves shown, but also, offers percentage data on the improvements resulting from the EBDT-TSCH mechanism. Additionally, in each subsection, the proposals presented in the manuscript are compared with the default TSCH configuration. To clarify that the proposed mechanism is compared with the 6TiSCH minimum default configuration, the term \textchange{``default minimal 6TiSCH configuration''} has been used in the new version. This term has also been updated in all Figures. 

We highlight that in addition to the comparison through simulation, we are explicit according to the contributions and challenges of our proposal.

\textchange{The above-related works mostly reduce node association time by increasing the beacon transmission rate. 
However, permanently increasing this rate increases energy consumption and traffic which could limit slots availability in the schedule and the communications between other network devices in networks with high node density. In our work, a dynamic beacons scheduling algorithm is proposed. In the intensive phase, the beacon transmission rate is high to accelerate the network formation. In the predetermined phase, this rate is decreased to avoid unnecessary traffic and corresponding energy consumption. The main contributions of our work are the following:}
\begin{itemize}
    \item \textchange{The proposed mechanism achieves a shorter network formation time than the default association mechanism (\textit{i.e.}, default minimal 6TiSCH configuration) without a relevant increase in implementation complexity.}
    \item \textchange{The proposal is accompanied by a mathematical model validated through simulation, showing a good correspondence between theoretical and simulation results.}
    \item \textchange{Simulation results demonstrated that the proposed mechanism not only reduces the network formation time resulting in data packets beginning to be transmitted earlier, but also reduces the energy consumption during the network formation process.}
\end{itemize}

Also, in the new manuscript version, before Subsection 4.1, we emphasize the following:

\textchange{``The proposed EBDT-TSCH does not imply a significant increase in implementation complexity...''}

%%%%

Finally, we have made several inclusions to the new version of the manuscript to guarantee its reproducibility. In the new version, we use the following footnote to describe the Contiki-NG used in the simulations:

\textchange{``The Contiki-NG version used for this research is the Develop v4.8, available at https://github.com/contiki-ng/contiki-ng/tree/release/v4.8.''}

A description of the hardware used to run the Cooja simulator was added in the first paragraph of Section 5:

\textchange{``A computer with $16$GB of RAM and an Intel Core i7-1185G7 processor was used to run $500$ independent simulations ensuring a maximum mean error of $10\%$ for a $95\%$ confidence interval.''}

Also, a table was included with the parameters used in the simulation:

\textchange{``The parameters used in the simulation are shown in Table \ref{SImPar}'' } 

\begin{table}[!h]
\centering
\scriptsize
\begin{tabular}{@{}lcc@{}}
\toprule
{\color[HTML]{3166FF} \textbf{Parameter}}                                                                             & {\color[HTML]{3166FF} \textbf{Symbol}}                                                       & {\color[HTML]{3166FF} \textbf{Value/Calculation}}                                 \\ \midrule
{\color[HTML]{3166FF} Slotframe length}                                                                               & {\color[HTML]{3166FF} $L$}                                                                     & {\color[HTML]{3166FF} $11$ slots}                                       \\
{\color[HTML]{3166FF} Number of netwok channels}                                                                      & {\color[HTML]{3166FF} $m$}                                                                     & {\color[HTML]{3166FF} $4$, $8$, $16$}                                       \\
{\color[HTML]{3166FF} Maximum time between beacons}                                                                   & {\color[HTML]{3166FF} $Teb$}                                                                   & {\color[HTML]{3166FF} $4$ s}                                            \\
{\color[HTML]{3166FF} Minimum time between beacons}                                                                   & {\color[HTML]{3166FF} $\rho Teb$}                                                & {\color[HTML]{3166FF} $0.75Teb$}                                        \\
{\color[HTML]{3166FF} \begin{tabular}[c]{@{}l@{}}Maximum time between beacons in the intensive phase\\\end{tabular}} & {\color[HTML]{3166FF} $Tebi$}                                         & {\color[HTML]{3166FF} $\alpha Teb$, $\alpha=0.5$}                                         \\
{\color[HTML]{3166FF} \begin{tabular}[c]{@{}l@{}}Minimum time between beacons in the intensive phase\\\end{tabular}} & {\color[HTML]{3166FF} $\rho Tebi$}                                         & {\color[HTML]{3166FF} $0.75Tebi$}                                         \\

{\color[HTML]{3166FF} \begin{tabular}[c]{@{}l@{}}Number of EBs sent at the intensive phase \\\end{tabular}}           & {\color[HTML]{3166FF} $u$} & {\color[HTML]{3166FF} $\lfloor \beta m\rfloor$,$0\leq \beta \leq 1.8$} \\
{\color[HTML]{3166FF} Join-seeker scan period}                                                                        & {\color[HTML]{3166FF} -}                                                                     & {\color[HTML]{3166FF} $\dfrac{Teb-\rho Teb}{2Lm}$}   \\           
\bottomrule
\end{tabular}
\caption{\textchange{Simulation parameters.}}
\label{SImPar}
\end{table}

In addition, we must point out that the developed mathematical model is explicit in Subsection 4.1 and Appendix A. To emphasize this, in the new version of the manuscript after equation (2) we include the following:

\textchange{``In the following, we develop an analytical model for the proposed EBDT-TSCH.''}



\item  \textbf{Reviewer comment:} \textit{Eq. (3): The equation is not correct because it formalizes the expected elapsed time. $t_j$ should be replaced with $E(t_j)$.}

\textbf{Answer:} We sincerely thank the reviewer for pointing out this error, which has already been corrected in the new version of the manuscript, in equations (3), (6), and (A.1). We also replaced the term ``elapsed time'' with \textchange{``average elapsed time''}.

\end{enumerate}
\end{document}
